% !TEX encoding = UTF-8 Unicode
\documentclass[a4paper,12pt]{extreport}
\usepackage[utf8]{inputenc}
\usepackage{color}
\usepackage[T2A]{fontenc}
\usepackage[intlimits]{amsmath}
\usepackage{amssymb}
\usepackage{dsfont}
\usepackage{bm}
\usepackage{indentfirst}
% \usepackage[1cm]{fullpage}
\usepackage{geometry}
\usepackage{soul}
\usepackage[normalem]{ulem}
\usepackage[english,russian]{babel}
\usepackage{hyperref}
\hypersetup{
    colorlinks=true,
    linkcolor=blue,
    filecolor=magenta,      
    urlcolor=cyan,
}
\usepackage{mathtools}
\usepackage{graphicx}
\usepackage{multicol}
\usepackage{float}
\graphicspath{{Pictures/}}

\linespread{1.15}                    % Полуторный интвервал (ГОСТ Р 7.0.11-2011, 5.3.6)
% \Large


% \renewcommand{\labelenumi}{(\alph{enumi})} % Use letters for enumerate
% \DeclareMathOperator{\Sample}{Sample}
\let\vaccent=\v % rename builtin command \v{} to \vaccent{}
\usepackage{enumerate}
\renewcommand{\v}[1]{\ensuremath{\mathbf{#1}}} % for vectors
\newcommand{\gv}[1]{\ensuremath{\mbox{\boldmath$ #1 $}}} 
% for vectors of Greek letters
\newcommand{\uv}[1]{\ensuremath{\mathbf{\hat{#1}}}} % for unit vector
\newcommand{\abs}[1]{\left| #1 \right|} % for absolute value
\newcommand{\avg}[1]{\left< #1 \right>} % for average
\let\underdot=\d % rename builtin command \d{} to \underdot{}
\renewcommand{\d}[2]{\frac{d #1}{d #2}} % for derivatives
\newcommand{\dd}[2]{\frac{d^2 #1}{d #2^2}} % for double derivatives
\newcommand{\pd}[2]{\frac{\partial #1}{\partial #2}} 
% for partial derivatives
\newcommand{\pdd}[2]{\frac{\partial^2 #1}{\partial #2^2}} 
% for double partial derivatives
\newcommand{\pdc}[3]{\left( \frac{\partial #1}{\partial #2}
 \right)_{#3}} % for thermodynamic partial derivatives
\newcommand{\ket}[1]{\left| #1 \right>} % for Dirac bras
\newcommand{\bra}[1]{\left< #1 \right|} % for Dirac kets
\newcommand{\braket}[2]{\left< #1 \vphantom{#2} \right|
 \left. #2 \vphantom{#1} \right>} % for Dirac brackets
\newcommand{\matrixel}[3]{\left< #1 \vphantom{#2#3} \right|
 #2 \left| #3 \vphantom{#1#2} \right>} % for Dirac matrix elements
\newcommand{\grad}[1]{\gv{\nabla} #1} % for gradient
\let\divsymb=\div % rename builtin command \div to \divsymb
\renewcommand{\div}[1]{\gv{\nabla} \cdot \v{#1}} % for divergence
\newcommand{\curl}[1]{\gv{\nabla} \times \v{#1}} % for curl
\let\baraccent=\= % rename builtin command \= to \baraccent
\renewcommand{\=}[1]{\stackrel{#1}{=}} % for putting numbers above =
\providecommand{\wave}[1]{\v{\tilde{#1}}}
\providecommand{\fr}{\frac}
\providecommand{\RR}{\mathbb{R}}
\providecommand{\NN}{\mathbb{N}}
\providecommand{\seq}{\subseteq}
\providecommand{\e}{\varepsilon}

% LaTeX internal stuff
\newcommand{\innermid}{\nonscript\;\delimsize\vert\nonscript\;}
\newcommand{\activatebar}{%
  \begingroup\lccode`\~=`\|
  \lowercase{\endgroup\let~}\innermid 
  \mathcode`|=\string"8000
}

% Probability theory
\newcommand{\Expect}{\mathop{{}\mathrm{E}}}
\newcommand{\Expectmore}{\operatorname{E}\expectarg}
\newcommand{\covmore}{\operatorname{cov}\expectarg}
\DeclarePairedDelimiterX{\expectarg}[1]{[}{]}{%
  \ifnum\currentgrouptype=16 \else\begingroup\fi
  \activatebar#1
  \ifnum\currentgrouptype=16 \else\endgroup\fi
}
\newcommand{\Proba}{\mathrm{P}}
\newcommand{\Var}{\mathop{{}\mathrm{V}}}

% Stochastic processes
\newcommand{\generaltime}{t \geqslant 0}
\newcommand{\discretetime}{t = 1,\,2,\,\ldots}
\newcommand{\conttime}{t \in \realline^{+}}
\newcommand{\finiteconttime}[1]{0 \leqslant t \leqslant #1}
\newcommand{\finitediscretetime}[1]{\running{t}{1}{#1}}

\newcommand{\newprocess}[1]{
	\ensuremath{
		#1 = \left(#1 _t\right)_{\generaltime}
	}
}
\newcommand{\newprocessd}[1]{
	\ensuremath{
		#1 = \left(#1 _t\right)_{\discretetime}
	}
}
\newcommand{\newprocessfc}[2]{
	\ensuremath{
		#1 = \left(#1 _t\right)_{\finiteconttime{#2}}
	}
}

\newcommand{\trajectory}[1]{\big\{#1_s,\,0\leqslant s \leqslant t\big\}}
\newcommand{\firsttime}[1]{\ensuremath{\inf \{\generaltime: #1\}}}

\newcommand{\filtration}[1]{\mathcal{#1}}
\newcommand{\filtrationprocess}[2]{\filtration{#1}^{#2}}
\newcommand{\filtrationflow}[1]{\newprocess{\filtration{F}}}

\newcommand{\fbm}{B^H}

\geometry{paper=a4paper,top=1cm,bottom=2cm,left=2cm,right=2cm}

\begin{document}


\begin{center}
    \begin{tabular}{|p{15.5cm}|}
        \hline
        \textbf{ФКН ВШЭ, 3 курс, 3 модуль}\\
        \begin{center} \Large Материалы к коллоквиуму
        \end{center}\\
        \textbf{Вероятностные модели и статистика случайных процессов, весна 2017}\\
        \hline
    \end{tabular}
\end{center}

% Время выдачи задания: 10 февраля (пятница), 12:10.

% Срок сдачи: \textcolor{blue}{\bf 10 февраля (пятница), 12:40.}

% \section*{Правила сдачи}

% \begin{enumerate}
% \item Работу необходимо сдать преподавателю на листе бумаги
% до дедлайна.
% \item Сверху листа укажите ваши фамилию и имя.
% \item \textbf{Каждая из задач имеет стоимость 1 балл.}
% Максимально допустимая оценка за работу -- 10 баллов. 
% \end{enumerate}


\section*{\centeringТеоретический минимум}


\begin{enumerate}
	\item Сформулируйте определение случайного процесса как случайной функции.
	\item Сформулируйте определение сечения случайного процесса.
	\item Сформулируйте определение траектории случайного процесса.

	\item Сформулируйте определение случайного процесса с непрерывным временем.
	\item Сформулируйте определение случайного процесса с дискретным временем.
	\item Сформулируйте определение случайного поля.
	\item Сформулируйте определение векторнозначного случайного процесса.

	\item Приведите пример случайного процесса с непрерывным временем.
	\item Приведите пример случайного процесса с дискретным временем.
	\item Приведите пример случайного поля.
	\item Приведите пример векторнозначного случайного процесса.

	\item Сформулируйте определение семейства конечномерных распределений случайного процесса.
	\item Приведите пример функции, задающей конечномерные распределения случайного процесса.

	\item Сформулируйте определение математического ожидания случайного процесса.
	\item Сформулируйте определение дисперсии случайного процесса.
	\item Сформулируйте определение ковариационной функции случайного процесса.
	
	\item Сформулируйте определение непрерывного в среднем квадратичном случайного процесса.
	\item Сформулируйте определение случайного процесса с непрерывными траекториями.
	\item Сформулируйте определение стохастически непрерывного случайного процесса.

	\item Сформулируйте определение гауссовской случайной величины.
	\item Сформулируйте определение гауссовского случайного вектора.
	\item Приведите пример некоррелированных, но зависимых случайных величин.
	\item Запишите выражение для характеристической функции гауссовской
	случайной величины.
	\item Сформулируйте необходимые и достаточные условия гауссовости случайного вектора.

	\item Сформулируйте основное утверждение теоремы о нормальной корреляции
	в случае пары гауссовских случайных величин.
	\item Сформулируйте основное утверждение теоремы о нормальной корреляции
	в случае пары гауссовских случайных векторов.

	\item Сформулируйте определение винеровского процесса.
	\item Сформулируйте определение гауссовского процесса.
	\item Приведите пример гауссовского процесса.
	\item Сформулируйте определение процесса Орнштейна-Уленбека.

	\item Сформулируйте определение последовательности независимых 
	одинаково распределенных случайных величин.
	\item Сформулируйте определение сильно стационарного случайного процесса.
	\item Сформулируйте определение ковариационно стационарного случайного процесса.
	\item Приведите пример сильно стационарного случайного процесса.
	\item Приведите пример ковариационно стационарного случайного процесса.
	\item Сформулируйте свойства ковариационной функции 
	слабо стационарного случайного процесса.
	\item Сформулируйте определение случайного процесса,
	эргодичного в среднем квадратичном по математическому ожиданию.
	\item Приведите пример процесса, являющегося эргодичным по математическому ожиданию
	в среднем квадратичном.
	\item Приведите пример процесса, являющегося сильно стационарным, 
	но не эргодичным по математическому ожиданию
	в среднем квадратичном.
	\item Сформулируйте необходимые и достаточные условия эргодичности
	случайного процесса в среднем по математическому ожиданию.
	\item Сформулируйте необходимые и достаточные условия эргодичности
	слабо стационарного случайного процесса в среднем по математическому ожиданию.

	\item Сформулируйте определение процесса восстановления.
	\item Сформулируйте определение пуассоновского процесса.
	\item Сформулируйте определение процесса.
	\item Запишите выражение для математического ожидания
	однородного пуассоновского процесса с интенсивностью $\lambda > 0$.
	\item Запишите выражение для распределения сечения
	однородного пуассоновского процесса с интенсивностью $\lambda > 0$
	в момент $t$.
	\item Сформулируйте основные свойства приращений пуассоновского процесса.

	\item Сформулируйте определение приращений случайного процесса.
	\item Сформулируйте определение процесса с независимыми приращениями.
	\item Сформулируйте определение процесса со стационарными приращениями.

	\item Сформулируйте определение дискретной марковской цепи.
	\item Сформулируйте определение марковского свойства.
	\item Сформулируйте теорему о вероятности последовательности
	состояний дискретной марковской цепи.
	\item Сформулируйте теорему о вероятности перехода однородной дискретной марковской цепи
	в заданное состояние за $n$ шагов.

	\item Дайте определения существенного и несущественного состояний марковской цепи.
	\item Дайте определения возвратного и невозвратного состояний марковской цепи.
	\item Дайте определение сообщающихся состояний марковской цепи.
	\item Дайте определение неприводимой дискретной марковской цепи.
	\item Дайте определения периодического и непериодического состояний марковской цепи.
	\item Сформулируйте утверждение о разбиении множества 
	состояний дискретной марковской цепи на классы сообщающихся состояний. Какое отношение
	существует между состояниями внутри каждого из таких классов?
	\item Приведите пример дискретной марковской цепи.
	\item Приведите пример дискретной марковской цепи с периодическими состояниями.
	\item Приведите пример дискретной марковской цепи с невозвратными состояниями.

	\item Сформулируйте определение дискретного случайного блуждания с дискретным временем.
	\item Сформулируйте условия возвратности некоторого состояния дискретной марковской цепи,
	используя матрицу перехода за один шаг.
	\item Сформулируйте условия возвратности некоторого состояния дискретной марковской цепи,
	используя вероятности вернуться в это состояние за конечное число шагов.

	\item Сформулируйте условия возвратности некоторого состояния дискретной марковской цепи,
	используя вероятности вернуться в это состояние за конечное число шагов.

	\item Сформулируйте определение эргодической марковской цепи.
	\item Сформулируйте определение стационарного распределения вероятностей
	дискретной марковской цепи.
	\item Сформулируйте первую эргодическую теорему для дискретной марковской цепи.

	\item Запишите выражение для авторегрессионной модели порядка $p$.
	\item Запишите выражение для модели скользящего среднего порядка $q$.
	\item Запишите выражение для смешанной модели авторегрессии и скользящего среднего порядков $(p, q)$.
	\item Запишите выражение для смешанной модели интегральной авторегрессии
	и скользящего среднего порядков $(p, d, q)$.
	\item Запишите выражение для авторегрессионной модели условной неоднородности порядка $(p)$.
	\item Запишите выражение для обобщенной авторегрессионной модели
	условной неоднородности порядков $(p, q)$.
\end{enumerate}

\newpage

\section*{\centeringТеоретический максимум}

\begin{enumerate}
	\item Сформулировать определение семейства конечномерных распределений случайного процесса.
	Сформулировать теорему А.~Н.~Колмогорова о существовании семействе конечномерных
	распределений случайного процесса. С использованием теоремы А.~Н.~Колмогорова 
	продемонстрировать невозможность существования непрерывного случайного процесса
	с сечениями, являющимися последовательностью независимых случайных величин.

	\item Сформулировать определение непрерывного в среднем квадратичном случайного процесса.
	Сформулировать и доказать необходимые и достаточные условия непрерывности
	случайного процесса в среднем квадратичном.

	\item Сформулировать определение стохастически непрерывного случайного процесса.
	С использованием определения стохастически непрерывного процесса
	доказать, что свойства стохастической непрерывности и независимости 
	сечений случайного процесса (при близких значениях времени) являются несовместными.

	\item Сформулировать и доказать утверждение о необходимых
	и достаточных условиях гауссовости случайного вектора.

	\item Сформулировать и доказать теорему о нормальной корреляции
	в случае пары гауссовских случайных величин.

	\item Сформулировать и доказать теорему о нормальной корреляции
	в случае пары гауссовских случайных векторов.

	\item Сформулировать определения и свойства винеровского
	и гауссовского процессов. Привести примеры гауссовских процессов.
	Описать полный набор параметров, однозначно определяющих гауссовский 
	процесс, обосновать это описание.

	\item Сформулировать определение процесса Орнштейна-Уленбека.
	Получить явный аналитический вид системы конечномерных
	распределений процесса Орнштейна-Уленбека.

	\item Перечислить классы стационарности случайных процессов,
	описать связь между ними.
	Привести примеры процессов, относящихся к каждому классу.

	\item Сформулировать определения процесса восстановления
	и пуассовноского потока событий. Сформулировать и доказать свойства
	(распределение, матожидание и дисперсию) пуассоновского
	потока событий с интенсивностью $\lambda > 0$ в момент $t$.

	\item Сформулировать свойства приращений пуассоновского процесса. 
	Получить явный аналитический вид системы конечномерных
	распределений пуассоновского процесса.

	\item Сформулировать определения марковского свойства и дискретной марковской цепи.
	Сформулировать и доказать теорему о вероятности последовательности
	состояний дискретной марковской цепи.

	\item Сформулировать определения марковского свойства и дискретной марковской цепи.
	Сформулировать и доказать теорему о вероятности перехода за $n$ шагов.

	\item Сформулировать классификацию состояний дискретной марковской цепи.

	\item Сформулировать определение дискретного случайного блуждания
	с дискретным временем. Описать частные типы этого процесса
	(симметричный, с отражением, с поглощением), описать типы
	его состояний. Сформулировать и доказать условия возвратности либо
	невозвратности случайного блуждания.

	\item Сформулировать определения возвратного и невозвратного состояний
	дискретной марковской цепи. Сформулировать и доказать утверждение
	о том, что возвратность или невозвратность состояния дискретной марковской цепи 
	следует из равенства или неравенства бесконечности
	величины $\sum\limits_{i=1}^n p^{(n)}_{ii}$, соответственно.

	\item Сформулировать определения возвратного и невозвратного состояний
	дискретной марковской цепи. Сформулировать и доказать утверждение
	о том, что возвратность или невозвратность состояния дискретной марковской цепи 
	равносильна тому, что вероятность $f_i$ события $\{\exists n \in \mathbb{N}: X_n = i\}$,
	где $n$ -- некоторый момент времени, $i$ -- рассматриваемое состояние,
	равняется либо меньше единицы, соответственно.

	\item Сформулировать определения возвратного, нулевого и периодического
	состояний дискретной марковской цепи. Сформулировать и доказать утверждение
	о том, что (1) если одно из состояний цепи нулевое, то и все остальные нулевые,
	(2) если одно из состояний возвратное, то и все остальные возвратные,
	(3) если одно из состояний периодическое с периодом $d$, то и все остальные
	периодические с периодом $d$.

	\item Вывести формулу средней длительности пребывания дискретной марковской цепи
	в заданном состоянии.

	\item Описать вычислительную разностную схему, позволяющую
	сгенерировать реализацию гауссовского случайного процесса 
	с помощью стохастического интегрирования по броуновскому движению
	(на примере процесса Орнштейна-Уленбека).

	\item Описать вычислительную схему, позволяющую
	сгенерировать реализацию гауссовского случайного процесса 
	с помощью разложения Холецкого
	(на примере процесса фрактального броуновского движения).

	\item Описать вычислительную схему, позволяющую
	сгенерировать реализацию однородного пуассоновского случайного процесса.

	\item Описать вычислительную схему, позволяющую
	сгенерировать реализацию неоднородного пуассоновского случайного процесса.

\end{enumerate}

\newpage

\section*{\centeringЗадачи}

\begin{enumerate}
	\item Подсчитайте математическое ожидание, дисперсию
	и ковариационную функцию случайного процесса
	$\newprocess{Y}$, задаваемого соотношением
	\[
	Y_t = a(t) X_t + b(t),
	\]
	где $\newprocess{X}$ -- случайный процесс с математическим ожиданием 
	$m(t) = \Expect X_t$, дисперсией $\sigma^2(t) = \Expectmore{X_t - \Expect X_t}^2$
	и ковариационной функцией $R(t_1, t_2) = 
	\Expectmore{(X_{t_1} - \Expect X_{t_1})(X_{t_2} - \Expect X_{t_2})}$.

	\item Доказать, что пуассоновский поток событий является стохастически 
	непрерывным случайным процессом.

	\item Пусть $\bm{\xi} = (\xi_1, \xi_2)^{\intercal}$ -- гауссовский случайный вектор
	с математическим ожиданием $\Expect \bm{\xi} = \bm{\mu} = (\mu_1, \mu_2)^{\intercal}$
	и ковариационной матрицей 
	\[
	\Expectmore{(\bm{\xi} - \bm{\mu})(\bm{\xi} - \bm{\mu})^{\intercal}} = 
	  \begin{bmatrix}
	    \sigma_1^2 & \rho \sigma_1 \sigma_2\\
	    \rho \sigma_1 \sigma_2 & \sigma_2^2
	  \end{bmatrix}.
	\]
	Выписать явное аналитическое выражение для двумерной плотности
	распределения случайного вектора $\bm{\xi}$.

	\item Пусть $\bm{\xi} = (\xi_1, \xi_2)^{\intercal}$ -- гауссовский случайный вектор
	с математическим ожиданием $\Expect \bm{\xi} = \bm{\mu} = (\mu_1, \mu_2)^{\intercal}$
	и ковариационной матрицей 
	\[
	\Expectmore{(\bm{\xi} - \bm{\mu})(\bm{\xi} - \bm{\mu})^{\intercal}} = 
	  \begin{bmatrix}
	    \sigma_1^2 & \rho \sigma_1 \sigma_2\\
	    \rho \sigma_1 \sigma_2 & \sigma_2^2
	  \end{bmatrix}.
	\]
	Подсчитать явное аналитическое выражение для условной плотности
	$f_{\xi_2|\xi_1}(x_2 | x_1)$
	распределения случайного вектора $\xi_2$ при условии $\xi_1 = x_1$.

	\item Пусть $N^1, N^2, \ldots, N^n$ -- независимые пуассоновские потоки событий
	с интенсивностями $\lambda_1, \lambda_2, \ldots, \lambda_n$, соответственно.
	Определить тип и параметры процесса $N_t = \sum\limits_{i=1}^n N^i_t$.

	\item Модель системы массового обслуживания, рассмотренная на лекции.
\end{enumerate}


\end{document}
