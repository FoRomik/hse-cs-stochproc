% !TEX encoding = UTF-8 Unicode
\documentclass[a4paper,14pt]{extreport}
\usepackage[utf8]{inputenc}
\usepackage{color}
\usepackage[T2A]{fontenc}
\usepackage[intlimits]{amsmath}
\usepackage{amssymb}
\usepackage{dsfont}
\usepackage{bm}
\usepackage{indentfirst}
\usepackage{fullpage}
\usepackage{soul}
\usepackage[normalem]{ulem}
\usepackage[english,russian]{babel}
\usepackage{hyperref}
\hypersetup{
    colorlinks=true,
    linkcolor=blue,
    filecolor=magenta,      
    urlcolor=cyan,
}
\usepackage{mathtools}
\usepackage{graphicx}
\graphicspath{{Pictures/}}

\linespread{1.33}                    % Полуторный интвервал (ГОСТ Р 7.0.11-2011, 5.3.6)
% \Large


% \renewcommand{\labelenumi}{(\alph{enumi})} % Use letters for enumerate
% \DeclareMathOperator{\Sample}{Sample}
\let\vaccent=\v % rename builtin command \v{} to \vaccent{}
\usepackage{enumerate}
\renewcommand{\v}[1]{\ensuremath{\mathbf{#1}}} % for vectors
\newcommand{\gv}[1]{\ensuremath{\mbox{\boldmath$ #1 $}}} 
% for vectors of Greek letters
\newcommand{\uv}[1]{\ensuremath{\mathbf{\hat{#1}}}} % for unit vector
\newcommand{\abs}[1]{\left| #1 \right|} % for absolute value
\newcommand{\avg}[1]{\left< #1 \right>} % for average
\let\underdot=\d % rename builtin command \d{} to \underdot{}
\renewcommand{\d}[2]{\frac{d #1}{d #2}} % for derivatives
\newcommand{\dd}[2]{\frac{d^2 #1}{d #2^2}} % for double derivatives
\newcommand{\pd}[2]{\frac{\partial #1}{\partial #2}} 
% for partial derivatives
\newcommand{\pdd}[2]{\frac{\partial^2 #1}{\partial #2^2}} 
% for double partial derivatives
\newcommand{\pdc}[3]{\left( \frac{\partial #1}{\partial #2}
 \right)_{#3}} % for thermodynamic partial derivatives
\newcommand{\ket}[1]{\left| #1 \right>} % for Dirac bras
\newcommand{\bra}[1]{\left< #1 \right|} % for Dirac kets
\newcommand{\braket}[2]{\left< #1 \vphantom{#2} \right|
 \left. #2 \vphantom{#1} \right>} % for Dirac brackets
\newcommand{\matrixel}[3]{\left< #1 \vphantom{#2#3} \right|
 #2 \left| #3 \vphantom{#1#2} \right>} % for Dirac matrix elements
\newcommand{\grad}[1]{\gv{\nabla} #1} % for gradient
\let\divsymb=\div % rename builtin command \div to \divsymb
\renewcommand{\div}[1]{\gv{\nabla} \cdot \v{#1}} % for divergence
\newcommand{\curl}[1]{\gv{\nabla} \times \v{#1}} % for curl
\let\baraccent=\= % rename builtin command \= to \baraccent
\renewcommand{\=}[1]{\stackrel{#1}{=}} % for putting numbers above =
\providecommand{\wave}[1]{\v{\tilde{#1}}}
\providecommand{\fr}{\frac}
\providecommand{\RR}{\mathbb{R}}
\providecommand{\NN}{\mathbb{N}}
\providecommand{\seq}{\subseteq}
\providecommand{\e}{\varepsilon}

% LaTeX internal stuff
\newcommand{\innermid}{\nonscript\;\delimsize\vert\nonscript\;}
\newcommand{\activatebar}{%
  \begingroup\lccode`\~=`\|
  \lowercase{\endgroup\let~}\innermid 
  \mathcode`|=\string"8000
}

% Probability theory
\newcommand{\Expect}{\mathop{{}\mathrm{E}}}
\newcommand{\Expectmore}{\operatorname{E}\expectarg}
\newcommand{\covmore}{\operatorname{cov}\expectarg}
\DeclarePairedDelimiterX{\expectarg}[1]{[}{]}{%
  \ifnum\currentgrouptype=16 \else\begingroup\fi
  \activatebar#1
  \ifnum\currentgrouptype=16 \else\endgroup\fi
}
\newcommand{\Proba}{\mathrm{P}}
\newcommand{\Var}{\mathop{{}\mathrm{V}}}

% Stochastic processes
\newcommand{\generaltime}{t \geqslant 0}
\newcommand{\discretetime}{n = 1,\,2,\,\ldots}
\newcommand{\conttime}{t \in \realline^{+}}
\newcommand{\finiteconttime}[1]{0 \leqslant t \leqslant #1}
\newcommand{\finitediscretetime}[1]{\running{t}{1}{#1}}

\newcommand{\newprocess}[1]{
    \ensuremath{
        #1 = \left(#1 _t\right)_{\generaltime}
    }
}
\newcommand{\newprocessd}[1]{
    \ensuremath{
        #1 = \left(#1 _n\right)_{\discretetime}
    }
}
\newcommand{\newprocessfc}[2]{
    \ensuremath{
        #1 = \left(#1 _t\right)_{\finiteconttime{#2}}
    }
}

\newcommand{\trajectory}[1]{\big\{#1_s,\,0\leqslant s \leqslant t\big\}}
\newcommand{\firsttime}[1]{\ensuremath{\inf \{\generaltime: #1\}}}

\newcommand{\filtration}[1]{\mathcal{#1}}
\newcommand{\filtrationprocess}[2]{\filtration{#1}^{#2}}
\newcommand{\filtrationflow}[1]{\newprocess{\filtration{F}}}

\newcommand{\fbm}{B^H}

\newenvironment{boenumerate}
  {\begin{enumerate}\renewcommand\labelenumi{\textbf{\theenumi.}}}
  {\end{enumerate}}


\begin{document}


\begin{center}
    \begin{tabular}{|p{15.5cm}|}
        \hline
        \textbf{ФКН ВШЭ, 3 курс, 3 модуль}\\
        \begin{center} \Large Задание 3. Статистические решения. Последовательные тесты.\\
        Обнаружение разладки.
        \end{center}\\
        \textbf{Вероятностные модели и статистика случайных процессов, весна 2017}\\
        \hline
    \end{tabular}
\end{center}

Время выдачи задания: 15 марта (среда).

Срок сдачи: \textcolor{blue}{\bf 27 марта (понедельник), 23:59.}

Среда для выполнения практического задания~-- PYTHON 2.x.

\section*{Правила сдачи}

\textbf{Инструкция по отправке:} 
\begin{enumerate}
\item Домашнее задание необходимо отправить
до дедлайна на почту \href{mailto:hse.cs.stochastics@gmail.com}
{hse.cs.stochastics@gmail.com}.
\item В письме укажите тему 
<<[ФКН ССП17] Задание 3, Фамилия Имя>>.
\item Решения задач следует присылать единым файлом
формата .pdf, набранным в \LaTeX. Допускается отправка
отдельных практических задач в виде отдельных файлов (ipython-тетрадок
или исходных файлов с кодом на языке \texttt{python}).
\end{enumerate}

\textbf{Оценивание и штрафы:} 
\begin{enumerate}
\item \textbf{Каждая из задач имеет стоимость 2 балла}, при этом
за задачу можно получить 0, 1 или 2 балла.
Максимально допустимая оценка за работу -- 10 баллов. 
Баллы, набранные сверх максимальной оценки, 
считаются бонусными и влияют на освобождение от задач на экзамене.
\item Дедлайн жесткий. Сдавать задание после указанного
срока сдачи нельзя.
\item Задание выполняется самостоятельно.
<<Похожие>> решения считаются плагиатом и все задействованные
студенты (в том числе те, у кого списали) не могут получить
за него больше 0 баллов (подробнее о плагиате см. на странице курса).
Если вы нашли решение какого-то из заданий (или его часть)
в открытом источнике, необходимо указать ссылку на этот источник
в отдельном блоке в конце Вашей работы (скорее всего вы будете не
единственным, кто это нашел, поэтому чтобы исключить подозрение
в плагиате, необходима ссылка на источник).
\end{enumerate}


\newpage

\section*{Необходимые теоретические сведения}

\begin{boenumerate}
    \item Всюду в рассматриваемых задачах
    имеется две гипотезы $\mathbb{H}_0$ и $\mathbb{H}_1$ 
    (иногда они обозначаются $\mathbb{H}_{\infty}$ и
    $\mathbb{H}_0$, соответственно), причем каждая 
    из гипотез делает явные предположения о распределении
    или его параметрах.
    \item Критерий Неймана-Пирсона предписывает
    принимать гипотезу исходя из значения величины
    \[
    L_n(X_1, \ldots, X_n) = 
        \frac
        {f_1(X_1, \ldots, X_n)}
        {f_0(X_1, \ldots, X_n)},
    \]
    называемой отношением правдоподобия.
    А именно, пусть $\varphi(X_1, \ldots, X_n)$~-- 
    рандомизированное решающее правило, значение
    которого равно вероятности принять 
    гипотезу $\mathbb{H}_1$. Тогда найдутся такие
    константы $\lambda_a$ и $h_a$, что 
    \[
    \varphi(X_1, \ldots, X_n) = 
    \begin{cases}
        1, & L_n(X_1, \ldots, X_n) > h_a, \\
        \lambda_a, & L_n(X_1, \ldots, X_n) = h_a, \\
        0, & L_n(X_1, \ldots, X_n) < h_a,
    \end{cases}
    \]
    является наиболее мощным 
    (т.\,е. с наименьшей вероятностью пропуска цели 
    или ошибки 2 рода $\beta(\varphi)$) тестом 
    среди тестов, вероятность ложной
    тревоги $\alpha(\varphi)$
    (ошибки 1 рода) которых не выше~$a$.

    \item Последовательный тест отношения
    правдоподобия (sequential pro-bability 
    ratio test, SPRT) заключается в вычислении
    логарифма отношения правдоподобия $Z_n = \log L_n$
    (см. выше; в случае независимых наблюдений
    формулы упрощаются) и сравнении этой величины
    в каждый момент времени с пороговыми
    значениями $A < 0, B > 0$, выбранными исходя
    из заданных вероятностей ошибок 1 и 2 рода.
    Наблюдения останавливаются в первый
    момент времени выхода статистики 
    $Z_n$ за <<коридор>> $(A, B)$: 
    \[
    \tau_{A,B} = \inf \{n \geqslant 1: 
    Z_n \notin (A, B)\}. 
    \]
    При этом в каждый момент времени принимается
    одно из трех решений: 
    \[
    \begin{cases}
        \text{если } Z_n \leqslant A & \implies
            \text{верна гипотеза } \mathbb{H}_0, \\
        \text{если } Z_n \geqslant B & \implies
            \text{верна гипотеза } \mathbb{H}_1, \\
        \text{если } Z_n \in (A, B) & \implies
            \text{продолжить наблюдения}.
    \end{cases}
    \]

    Построить последовательный тест~-- значит
    указать \textit{момент остановки измерений $\tau$}
    и \textit{решающее правило $\varphi(\cdot)$}.

    \item Разладкой процесса $\newprocessd{X}$
    называется ситуация, в которой траектория
    процесса генерируется двумя (или 
    в общем случае несколькими) независимыми
    вероятностными мерами $\Proba_{\infty}$
    и $\Proba_{0}$, причем наблюдения имеют структуру
    \[
    X_n = 
    \begin{cases}
        X^{\infty}_n, & \text{если } 1 \leqslant n < \theta, \\
        X^{0}_n, & \text{если } n \geqslant \theta,
    \end{cases}
    \]
    где $\newprocessd{X^{\infty}}$ --- 
    процесс, соответствующий мере $\Proba_{\infty}$,
    и $\newprocessd{X^{0}}$ ---
    процесс, соответствующий мере $\Proba_{0}$.
    Момент $\theta \in [0, \infty]$
    называется моментом разладки, причем
    ситуация $\theta = 0$ соответствует тому, что 
    с самого начала идут наблюдения от 
    <<разлаженного>> процесса $X^{0}$,
    а ситуация $\theta = \infty$ заключается в том,
    что разладка не появляется никогда.
    Таким образом, траектория процесса $X$ выглядит
    следующим образом:
    \[
    \underbrace{X^{\infty}_1, X^{\infty}_2, \ldots, 
    X^{\infty}_{\theta - 1}}
        _{\text{мера } \Proba^{\infty}},
    \underbrace{X^{0}_{\theta}, X^{0}_{\theta + 1}, \ldots}
        _{\text{мера } \Proba^{0}}
    \]

    \item \textbf{Статистика кумулятивных сумм.}
    \begin{itemize}
    \item Вводятся статистики $\newprocessd{\gamma}$ 
     и $\newprocessd{\gamma}$ 
    \begin{equation*}
    \gamma_n = \sup\limits_{\theta \geqslant 0} \frac{f_{\theta}(X_1, \ldots, X_n)}{f_{\infty}(X_1, \ldots, X_n)}
    \qquad \mbox{и} \qquad T_n = \log \gamma_n
    \end{equation*}
    \item Если случайные величины
    $X_1, \ldots, X_n$ независимы, то
    \begin{align*}
    \gamma_n &= \max\Big\{1, \max\limits_{1 \leqslant \theta \leqslant n} \prod\limits_{k=\theta}^{n}
        \frac{f_{0}(X_k)}{f_{\infty}(X_k)} \Big\},\\
    T_n &= \max\Big\{0, \max\limits_{1 \leqslant \theta \leqslant n} \sum\limits_{k=\theta}^{n}
        \log \frac{f_{0}(X_k)}{f_{\infty}(X_k)} \Big\} = 
        \max\Big\{0, \max\limits_{1 \leqslant \theta \leqslant n} \sum\limits_{k=\theta}^{n}
        \zeta_k \Big\}
    \end{align*}
    \item Статистика $T_n$ обладает свойством $T_n = \max (0, T_{n-1} + \zeta_n)$
    и называется статистикой кумулятивных сумм (CUmulative SUMs, CUSUM).
    \item Момент остановки
    \[
    \tau_{\mathrm{CUSUM}} = \inf \{n \geqslant 0: T_n \geqslant B\},
    \]
    построенный по статистике кумулятивных сумм,
    оптимален (т.\,е.~обладает наименьшей задержкой
    в обнаружении разладки) в~классе
    \[
    \mathcal{M}_T = \{\tau : 
        {\textstyle \Expect_{\infty}} \tau \geqslant T\}
    \]
    тех моментов остановки, для которых среднее время
    до~ложной тревоги не меньше $T$.

    \end{itemize}


    \item \textbf{Статистика Ширяева-Робертса.}
    \begin{itemize}
    \item Вводится статистика
    \begin{equation*}
    R_n = \sum\limits_{\theta = 1}^{n} \frac{f_{\theta}(X_1, \ldots, X_n)}{f_{\infty}(X_1, \ldots, X_n)}
    \end{equation*}
    \item Если случайные величины $X_1, \ldots, X_n$ независимы, то
    \begin{equation*}
    R_n = \sum\limits_{\theta = 1}^{n} \prod\limits_{k=\theta}^{n}
        \frac{f_{0}(X_k)}{f_{\infty}(X_k)} =
            \sum\limits_{\theta = 1}^{n} \prod\limits_{k=\theta}^{n}
        l_k.
    \end{equation*}
    \item Статистика $R_n$ обладает свойством $R_n = (1 + R_{n-1}) l_k$
    и~называется статистикой Ширяева-Робертса (Shiryaev-Roberts, SR).
    \item Момент остановки 
    \begin{equation*}
    \tau_{\mathrm{SR}} = \inf \{n \geqslant 0: R_n \geqslant B\},
    \end{equation*}
    построенный по статистике Ширяева-Робертса,
    оптимален (т.\,е.~обладает наименьшей задержкой
    в обнаружении разладки) в~классе
    \[
    \mathcal{M}_T = \{\tau : 
        {\textstyle \Expect_{\infty}} \tau \geqslant T\}
    \]
    тех моментов остановки, для которых среднее время
    до~ложной тревоги не меньше $T$.
    \end{itemize}


\end{boenumerate}


\newpage


\section*{Вариант 1}


\begin{enumerate}

    \item По выборке $(X_1, \ldots, X_n)$ из
    биномиального распределения $\text{Bin}(k, p)$
    построить критерий Неймана-Пирсона 
    для проверки гипотезы $\mathbb{H}_0: p = p_0$
    против альтернативы $\mathbb{H}_1: p = p_1$,
    где $0 < p_0 < p_1 < 1$.

    \item Дана выборка $(X_1, \ldots, X_n)$ из
    нормального $\mathcal{N}(\mu, \sigma^2)$
    распределения. Построить
    критерий проверки гипотезы $\mathbb{H}_0: \mu = 0$
    против альтернативы $\mathbb{H}_1: \mu = 0.1$
    и определить наименьший объем выборки, 
    при котором вероятности ошибок 1 и 2 родов
    не превышают $0.01$. 

    \item Необходимо произвести выбор между
    двумя гипотезами о возможных значениях $p_0$
    и $p_1$ вероятности события $A$ ($p_0 < p_1$).
    В этих целях осуществляется последовательность
    независимых опытов, в каждом из которых
    определяется, происходит или не происходит
    событие $A$. Построить последовательный
    критерий отношения вероятностей при заданных
    значениях $\alpha$ и $\beta$ вероятностей
    ошибок первого и второго рода.

    
    \item Провести моделирование для сравнения
    критерия Неймана-Пирсона и последовательного 
    критерия отношения правдоподобия в задаче~3.
    В этом моделировании:
    \begin{enumerate}
        \item Для заданных уровня значимости
        $\alpha_i = i\Delta,
        \Delta = 0.01, i = 1, \ldots, 99,$
        и вероятности ошибки второго рода 
        $\beta_i = \alpha_i$ подсчитать 
        объем наблюдений, требуемый в критерии 
        Неймана-Пирсона для достижения этих
        характеристик.

        \item Проделать то же самое для 
        последовательного критерия отношения
        правдоподобия.

        \item Привести графическое сравнение
        зависимости объема требуемых данных
        от требуемого уровня значимости 
        $n(\alpha)$ для двух критериев, сделать
        выводы.

        \item Изменяется ли соотношение 
        между требуемыми объемами выборок
        при изменении отношения $\gamma = p_0 / p_1$
        в рассматриваемых гипотезах? 
        Построить зависимости $n(\gamma)$
        для двух критериев при некотором фиксированном
        уровне значимости $\alpha$.
    \end{enumerate}

    \item Процесс $\newprocessd{X}$, наблюдаемый
    в режиме реального времени, задается
    нормально распределенным белым шумом
    % $\newprocessd{\varepsilon}$ 
    (с нулевым  средним и единичной дисперсией), 
    т.\,е. 
    \[
    X_n = \varepsilon_n, \qquad n=1, 2, \ldots
    \]
    В неизвестный момент времени $\theta \geqslant 1$
    происходит разладка 
    (изменение статистических свойств) процесса
    $X_n$, которая состоит в том, что 
    для $n \geqslant \theta$ процесс $X$
    задается уравнением типа AR(1), то есть
    \[
    X_n = \alpha_0 + \alpha_1 X_{n-1} + \varepsilon_n,
    \qquad n \geqslant \theta,
    \]
    где $|\alpha_1| < 1$. 

    Построить процедуру обнаружения
    разладки, основанную на статистике кумулятивных
    сумм, для обнаружения момента $\theta$.
    Параметры $\alpha_0, \alpha_1$
    процесса считать известными. 
    Привести формулы для отношения правдоподобия,
    а также для одного шага итеративного алгоритма
    кумулятивных сумм. В какой момент
    следует поднимать тревогу об обнаружении разладки?

    \item Провести моделирование для определения
    оперативных характеристик процедуры обнаружения
    разладки, разработанной в задаче~5. Считать
    заданными параметры $\alpha_0 = 0, \alpha_1 = 0.8$.
    \begin{enumerate}
        \item При использовании статистики 
        $\newprocessd{\gamma}$ прежде
        всего необходимо подобрать
        значение порога $B = B_T$ в зависимости
        от значения параметра $T$ так, чтобы
        $\tau (B_T ; \{\gamma_n\}) \in \mathcal{M}_T$.
        Требуется подсчитать (с помощью метода
        Монте-Карло) и дать в виде графика
        значения величины
        \[
        \mathbb{T}_{\mathrm{CUSUM}}(B) = 
        {\textstyle \Expect_{\infty}} \tau(B; \{\gamma_n\})
        \]
        для разных значений $B$ (и малых и больших).

        \item С помощью метода Монте-Карло подсчитать
        и дать в виде графика значения величины
        \[
        \mathbb{R}_{\text{CUSUM}}(B) = 
        {\textstyle \Expect_{0}} \tau(B; \{\gamma_n\}).
        \]
        для разных значений $B$ (и малых и больших). Графики нарисовать для достаточно
        частых значений $B$.

    \end{enumerate}

    \item Вам выданы файлы \texttt{sig1.train}
    (обучающий) и \texttt{sig1.test.public}
    (валидационный) (третий файл
    \texttt{sig1.test.private} имеется у лектора). 
    Обучающий файл содержит два столбца, причем
    первый столбец --- это реализация 
    $X_1, \ldots, X_{1000}$ некоторого
    случайного процесса, полученная следующим
    образом:
    \[
    X_n = 
    \begin{cases}
        X^{\infty}_n, & \text{если } n \notin [\theta, \theta + \Delta], \\
        X^{0}_n, & \text{если } n \in [\theta, \theta + \Delta],
    \end{cases}
    \]
    а второй столбец --- это индикатор действия
    процесса $X^{0}_n$, т.\,.е. процесс
    \[
    Y_n = \mathds{1}_{[\theta, \theta + \Delta]}(n) = 
    \begin{cases}
        0, & \text{если } n \notin [\theta, \theta + \Delta], \\
        1, & \text{если } n \in [\theta, \theta + \Delta].
    \end{cases}
    \]
    Сечения процесса $X$ могут быть как зависимы,
    так и независимы.

    \begin{enumerate}
        \item Предложите какие-либо
        модели временных рядов
        $X^{0}_n$ и $X^{\infty}_n$, 
        адекватно описывающие наблюдения
        обучающей выборки.

        \item Используя предложенные модели
        и рассмотренные на лекциях и семинарах
        подходы (полезно также рассматривать
        и их композиции), 
        предложите алгоритм обнаружения разладки
        процесса $X$. Этот алгоритм должен работать
        в режиме реального времени, т.\,е. для вынесения
        решения о разладке в момент $n$
        он не может использовать 
        всю доступную траекторию процесса
        $X$, а может использовать  лишь наблюдения
        до момента $n$ включительно.
        (Тем не менее, для построения алгоритма
        можно использовать все доступные данные).

        \item Реализуйте этот алгоритм в программном коде.

        \item Проверьте его работу на обучающих данных,
        нарисуйте траекторию статистики этого алгоритма,
        сравните ее с индикатором разладки.

        \item Нарисуйте траекторию статистики этого алгоритма на тестовых данных, вставьте
        в отчет рисунок. Сохраните эту траекторию
        в текстовый файл (по одному значению на строку)
        и пришлите вместе с исходным кодом,
        реализующим метод обнаружения разладки.
    \end{enumerate}

\end{enumerate}


\newpage 

\section*{Вариант 2}


\begin{enumerate}


    \item По выборке $(X_1, \ldots, X_n)$ из
    пуассоновского распределения $\Pi(\lambda)$
    построить критерий Неймана-Пирсона 
    для проверки гипотезы $\mathbb{H}_0: \lambda = \lambda_0$
    против альтернативы $\mathbb{H}_1: \lambda = \lambda_1$,
    где $0 < \lambda_0 < \lambda_1$.

    \item В последовательности $\xi_1, \ldots, \xi_n$
    независимых испытаний, выполненных согласно
    схеме Бернулли, $\Proba(\xi_i = 1) = p, 
    \Proba(\xi_i = 0) = 1 - p$. Построить
    критерий проверки гипотезы $\mathbb{H}_0: p = 0$
    против альтернативы $\mathbb{H}_1: p = 0.01$
    и определить наименьший объем выборки, 
    при котором вероятности ошибок 1 и 2 родов
    не превышают $0.01$.

    \item Пусть гипотезы $\mathbb{H}_0$ и $\mathbb{H}_1$ имеют вид
    \begin{eqnarray*}
    \mathbb{H}_0: &f(x) =
        \theta_0^{-1} \exp(-x/\theta_0), & \quad x > 0;\\
    \mathbb{H}_1: &f(x) = \theta_1^{-1} \exp(-x/\theta_1),& 
        \quad  x > 0, \quad \theta_1 = 2\theta_0;
    \end{eqnarray*}
    Построить процедуру последовательного критерия
    отношения правдоподобия различения
    гипотез $\mathbb{H}_0$ и $\mathbb{H}_1$
    при заданных величинах вероятностей ошибок
    первого и второго рода $\alpha = \beta \leqslant 0.05$.

    \item Провести моделирование для сравнения
    критерия Неймана-Пирсона и последовательного 
    критерия отношения правдоподобия в задаче~3.
    В этом моделировании:
    \begin{enumerate}
        \item Для заданных уровня значимости
        $\alpha_i = i\Delta,
        \Delta = 0.01, i = 1, \ldots, 99,$
        и вероятности ошибки второго рода 
        $\beta_i = \alpha_i$ подсчитать 
        объем наблюдений, требуемый в критерии 
        Неймана-Пирсона для достижения этих
        характеристик.

        \item Проделать то же самое для 
        последовательного критерия отношения
        правдоподобия.

        \item Привести графическое сравнение
        зависимости объема требуемых данных
        от требуемого уровня значимости 
        $n(\alpha)$ для двух критериев, сделать
        выводы.

        \item Изменяется ли соотношение 
        между требуемыми объемами выборок
        при изменении отношения $\gamma = \theta_0 / \theta_1$
        в рассматриваемых гипотезах? 
        Построить зависимости $n(\gamma)$
        для двух критериев при некотором фиксированном
        уровне значимости $\alpha$.
    \end{enumerate}


    \item Процесс $\newprocessd{X}$, наблюдаемый
    в режиме реального времени, задается
    нормально распределенным белым шумом
    % $\newprocessd{\varepsilon}$ 
    (с нулевым  средним и единичной дисперсией), 
    т.\,е. 
    \[
    X_n = \varepsilon_n, \qquad n=1, 2, \ldots
    \]
    В неизвестный момент времени $\theta \geqslant 1$
    происходит разладка 
    (изменение статистических свойств) процесса
    $X_n$, которая состоит в том, что 
    для $n \geqslant \theta$ процесс $X$
    задается уравнением типа ARCH(1), то есть
    \[
    X_n = \sigma_n \varepsilon_n,
    \qquad
    \sigma^2_n = \alpha_0 + \alpha_1 X^2_{n-1},
    \qquad n \geqslant \theta,
    \]
    где $|\alpha_1| < 1$. 

    Построить процедуру обнаружения
    разладки, основанную на статистике Ширяева-Робертса,
    для обнаружения момента $\theta$.
    Параметры $\alpha_0, \alpha_1$
    процесса считать известными. 
    Привести формулы для отношения правдоподобия,
    а также для одного шага итеративного алгоритма
    Ширяева-Робертса. В какой момент
    следует поднимать тревогу об обнаружении разладки?


    \item Провести моделирование для определения
    оперативных характеристик процедуры обнаружения
    разладки, разработанной в задаче~5. Считать
    заданными параметры $\alpha_0 = 0.146,
    \alpha_1 = 0.107$.
    \begin{enumerate}
        \item При использовании статистики 
        $\newprocessd{\gamma}$ прежде
        всего необходимо подобрать
        значение порога $B = B_T$ в зависимости
        от значения параметра $T$ так, чтобы
        $\tau (B_T ; \{\gamma_n\}) \in \mathcal{M}_T$.
        Требуется подсчитать (с помощью метода
        Монте-Карло) и дать в виде графика
        значения величины
        \[
        \mathbb{T}_{\mathrm{SR}}(B) = 
        {\textstyle \Expect_{\infty}} \tau(B; \{\gamma_n\})
        \]
        для разных значений $B$ (и малых и больших).

        \item С помощью метода Монте-Карло подсчитать
        и дать в виде графика значения величины
        \[
        \mathbb{R}_{\text{SR}}(B) = 
        {\textstyle \Expect_{0}} \tau(B; \{\gamma_n\}).
        \]
        для разных значений $B$ (и малых и больших). Графики нарисовать для достаточно
        частых значений $B$.

    \end{enumerate}

    \item Вам выданы файлы \texttt{sig2.train}
    (обучающий) и \texttt{sig2.test.public}
    (валидационный) (третий файл
    \texttt{sig2.test.private} имеется у лектора). 
    Обучающий файл содержит два столбца, причем
    первый столбец --- это реализация 
    $X_1, \ldots, X_{1000}$ некоторого
    случайного процесса, полученная следующим
    образом:
    \[
    X_n = 
    \begin{cases}
        X^{\infty}_n, & \text{если } n \notin [\theta, \theta + \Delta], \\
        X^{0}_n, & \text{если } n \in [\theta, \theta + \Delta],
    \end{cases}
    \]
    а второй столбец --- это индикатор действия
    процесса $X^{0}_n$, т.\,.е. процесс
    \[
    Y_n = \mathds{1}_{[\theta, \theta + \Delta]}(n) = 
    \begin{cases}
        0, & \text{если } n \notin [\theta, \theta + \Delta], \\
        1, & \text{если } n \in [\theta, \theta + \Delta].
    \end{cases}
    \]
    Сечения процесса $X$ могут быть как зависимы,
    так и независимы.

    \begin{enumerate}
        \item Предложите какие-либо
        модели временных рядов
        $X^{0}_n$ и $X^{\infty}_n$, 
        адекватно описывающие наблюдения
        обучающей выборки.

        \item Используя предложенные модели
        и рассмотренные на лекциях и семинарах
        подходы (полезно также рассматривать
        и их композиции), 
        предложите алгоритм обнаружения разладки
        процесса $X$. Этот алгоритм должен работать
        в режиме реального времени, т.\,е. для вынесения
        решения о разладке в момент $n$
        он не может использовать 
        всю доступную траекторию процесса
        $X$, а может использовать  лишь наблюдения
        до момента $n$ включительно.
        (Тем не менее, для построения алгоритма
        можно использовать все доступные данные).

        \item Реализуйте этот алгоритм в программном коде.

        \item Проверьте его работу на обучающих данных,
        нарисуйте траекторию статистики этого алгоритма,
        сравните ее с индикатором разладки.

        \item Нарисуйте траекторию статистики этого алгоритма на тестовых данных, вставьте
        в отчет рисунок. Сохраните эту траекторию
        в текстовый файл (по одному значению на строку)
        и пришлите вместе с исходным кодом,
        реализующим метод обнаружения разладки.
    \end{enumerate}


\end{enumerate}

\end{document}
