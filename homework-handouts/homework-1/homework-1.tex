% !TEX encoding = UTF-8 Unicode
\documentclass[a4paper,14pt]{extreport}
\usepackage[utf8]{inputenc}
\usepackage{color}
\usepackage[T2A]{fontenc}
\usepackage[intlimits]{amsmath}
\usepackage{amssymb}
\usepackage{dsfont}
\usepackage{bm}
\usepackage{indentfirst}
\usepackage{fullpage}
\usepackage{soul}
\usepackage[normalem]{ulem}
\usepackage[english,russian]{babel}
\usepackage{hyperref}
\hypersetup{
    colorlinks=true,
    linkcolor=blue,
    filecolor=magenta,      
    urlcolor=cyan,
}
\usepackage{mathtools}
\usepackage{graphicx}
\graphicspath{{Pictures/}}

\linespread{1.33}                    % Полуторный интвервал (ГОСТ Р 7.0.11-2011, 5.3.6)
% \Large


% \renewcommand{\labelenumi}{(\alph{enumi})} % Use letters for enumerate
% \DeclareMathOperator{\Sample}{Sample}
\let\vaccent=\v % rename builtin command \v{} to \vaccent{}
\usepackage{enumerate}
\renewcommand{\v}[1]{\ensuremath{\mathbf{#1}}} % for vectors
\newcommand{\gv}[1]{\ensuremath{\mbox{\boldmath$ #1 $}}} 
% for vectors of Greek letters
\newcommand{\uv}[1]{\ensuremath{\mathbf{\hat{#1}}}} % for unit vector
\newcommand{\abs}[1]{\left| #1 \right|} % for absolute value
\newcommand{\avg}[1]{\left< #1 \right>} % for average
\let\underdot=\d % rename builtin command \d{} to \underdot{}
\renewcommand{\d}[2]{\frac{d #1}{d #2}} % for derivatives
\newcommand{\dd}[2]{\frac{d^2 #1}{d #2^2}} % for double derivatives
\newcommand{\pd}[2]{\frac{\partial #1}{\partial #2}} 
% for partial derivatives
\newcommand{\pdd}[2]{\frac{\partial^2 #1}{\partial #2^2}} 
% for double partial derivatives
\newcommand{\pdc}[3]{\left( \frac{\partial #1}{\partial #2}
 \right)_{#3}} % for thermodynamic partial derivatives
\newcommand{\ket}[1]{\left| #1 \right>} % for Dirac bras
\newcommand{\bra}[1]{\left< #1 \right|} % for Dirac kets
\newcommand{\braket}[2]{\left< #1 \vphantom{#2} \right|
 \left. #2 \vphantom{#1} \right>} % for Dirac brackets
\newcommand{\matrixel}[3]{\left< #1 \vphantom{#2#3} \right|
 #2 \left| #3 \vphantom{#1#2} \right>} % for Dirac matrix elements
\newcommand{\grad}[1]{\gv{\nabla} #1} % for gradient
\let\divsymb=\div % rename builtin command \div to \divsymb
\renewcommand{\div}[1]{\gv{\nabla} \cdot \v{#1}} % for divergence
\newcommand{\curl}[1]{\gv{\nabla} \times \v{#1}} % for curl
\let\baraccent=\= % rename builtin command \= to \baraccent
\renewcommand{\=}[1]{\stackrel{#1}{=}} % for putting numbers above =
\providecommand{\wave}[1]{\v{\tilde{#1}}}
\providecommand{\fr}{\frac}
\providecommand{\RR}{\mathbb{R}}
\providecommand{\NN}{\mathbb{N}}
\providecommand{\seq}{\subseteq}
\providecommand{\e}{\varepsilon}

% LaTeX internal stuff
\newcommand{\innermid}{\nonscript\;\delimsize\vert\nonscript\;}
\newcommand{\activatebar}{%
  \begingroup\lccode`\~=`\|
  \lowercase{\endgroup\let~}\innermid 
  \mathcode`|=\string"8000
}

% Probability theory
\newcommand{\Expect}{\mathop{{}\mathrm{E}}}
\newcommand{\Expectmore}{\operatorname{E}\expectarg}
\newcommand{\covmore}{\operatorname{cov}\expectarg}
\DeclarePairedDelimiterX{\expectarg}[1]{[}{]}{%
  \ifnum\currentgrouptype=16 \else\begingroup\fi
  \activatebar#1
  \ifnum\currentgrouptype=16 \else\endgroup\fi
}
\newcommand{\Proba}{\mathrm{P}}
\newcommand{\Var}{\mathop{{}\mathrm{V}}}

% Stochastic processes
\newcommand{\generaltime}{t \geqslant 0}
\newcommand{\discretetime}{t = 1,\,2,\,\ldots}
\newcommand{\conttime}{t \in \realline^{+}}
\newcommand{\finiteconttime}[1]{0 \leqslant t \leqslant #1}
\newcommand{\finitediscretetime}[1]{\running{t}{1}{#1}}

\newcommand{\newprocess}[1]{
	\ensuremath{
		#1 = \left(#1 _t\right)_{\generaltime}
	}
}
\newcommand{\newprocessd}[1]{
	\ensuremath{
		#1 = \left(#1 _t\right)_{\discretetime}
	}
}
\newcommand{\newprocessfc}[2]{
	\ensuremath{
		#1 = \left(#1 _t\right)_{\finiteconttime{#2}}
	}
}

\newcommand{\trajectory}[1]{\big\{#1_s,\,0\leqslant s \leqslant t\big\}}
\newcommand{\firsttime}[1]{\ensuremath{\inf \{\generaltime: #1\}}}

\newcommand{\filtration}[1]{\mathcal{#1}}
\newcommand{\filtrationprocess}[2]{\filtration{#1}^{#2}}
\newcommand{\filtrationflow}[1]{\newprocess{\filtration{F}}}

\newcommand{\fbm}{B^H}

\begin{document}


\begin{center}
    \begin{tabular}{|p{15.5cm}|}
        \hline
        \textbf{ФКН ВШЭ, 3 курс, 3 модуль}\\
        \begin{center} \Large Задание 1. Основы теории случайных процессов
        \end{center}\\
        \textbf{Вероятностные модели и статистика случайных процессов, весна 2017}\\
        \hline
    \end{tabular}
\end{center}

Время выдачи задания: 23 января (понедельник).

Срок сдачи: \textcolor{blue}{\bf 5 февраля (воскресенье), 23:59.}

Среда для выполнения практического задания~-- PYTHON 2.x.

\section*{Правила сдачи}

\textbf{Инструкция по отправке:} 
\begin{enumerate}
\item Домашнее задание необходимо отправить
до дедлайна на почту \href{mailto:hse.cs.stochastics@gmail.com}
{hse.cs.stochastics@gmail.com}.
\item В письме укажите тему 
<<[ФКН ССП17] Задание 1, Фамилия Имя>>.
\item Решения задач следует присылать единым файлом
формата .pdf, набранным в \LaTeX. Допускается отправка
последней задачи в виде отдельной ipython-тетрадки.
\end{enumerate}

\textbf{Оценивание и штрафы:} 
\begin{enumerate}
\item \textbf{Каждая из задач имеет стоимость 2 балла}, при этом
за задачу можно получить 0, 1 или 2 балла.
Максимально допустимая оценка за работу -- 10 баллов. 
Баллы, набранные сверх максимальной оценки, 
считаются бонусными и влияют на освобождение от задач на экзамене.
\item Дедлайн жесткий. Сдавать задание после указанного
срока сдачи нельзя.
\item Задание выполняется самостоятельно.
<<Похожие>> решения считаются плагиатом и все задействованные студенты (в том числе те, у кого списали) не могут получить за него больше 0 баллов (подробнее о плагиате см. на странице курса). Если вы нашли решение какого-то из заданий (или его часть) в открытом источнике, необходимо указать ссылку на этот источник в отдельном блоке в конце Вашей работы (скорее всего вы будете не единственным, кто это нашел, поэтому чтобы исключить подозрение в плагиате, необходима ссылка на источник).
\end{enumerate}


\newpage

\section*{Необходимые теоретические сведения}

\textbf{1.} Ковариационной функцией $R_X(t_1, t_2)$ случайного процесса
$\newprocess{X}$ называется неслучайная функция
\[
R_X(t_1, t_2) = \mathrm{cov} (X_{t_1} - \Expect X_{t_1}) (X_{t_2} - \Expect X_{t_2}).
\]
Корреляционной функцией $r_X(t_1, t_2)$ случайного процесса $X$ 
называется неслучайная функция 
\[
r_X(t_1, t_2) = \frac{R_X(t_1, t_2)}{\sqrt{\Var X_{t_1} \Var X_{t_2}}},
\]
где $\Var X_t = \Expect (X_t - \Expect X_t)^2$ -- функция дисперсии
случайного процесса $X$.
Взаимной ковариационной функцией $R_{XY}(t_1, t_2)$ пары 
случайных процессов $\newprocess{X}$ и $\newprocess{Y}$
называется неслучайная функция 
\[
R_{XY}(t_1, t_2) = \mathrm{cov} (X_{t_1} - \Expect X_{t_1}) (Y_{t_2} - \Expect Y_{t_2}).
\]
Взаимная корреляционная функция $r_{XY}(t_1, t_2)$ определяется
аналогично равенству для $r_X(t_1, t_2)$ выше.

\textbf{2.} Случайный процесс $\newprocess{X}$ называется процессом
с независимыми приращениями, если для любых $t_0, t_1, \ldots, t_n$,
таких, что $0 = t_0 < t_1 < \ldots < t_n$ случайные величины
$X_1 - X_0, \ldots, X_{n} - X_{n-1}$ независимы в совокупности.

\textbf{3.} Вектор $X = (X_1, \dots, X_n)$ называется
гауссовским, если для любого набора коэффициентов
$(\lambda_1, \dots, \lambda_n) \in \mathbb{R}^n$
случайная величина $Y \stackrel{\text{def}}{=}
\sum_{k=1}^n \lambda_k X_k$ имеет нормальное распределение. 

\textbf{4.} Процесс $\newprocess{W}$ на $(\Omega, \mathcal{F}, \Proba)$
называется винеровским (или броуновским движением), если 
\begin{itemize}
    \item $W_0 = 0 \,\,\, \Proba$-п.н.,
    \item $W_t$ имеет независимые приращения $\forall t$,
    \item $W_t - W_s \sim \mathcal{N}(0, t- s) \,\,\, \forall t > s \ge 0$.
\end{itemize}


\newpage


\section*{Вариант 1}


\begin{enumerate}

\item Доказать, что данная функция может или не может
являться ковариационной функцией случайного процесса:
	\begin{enumerate}
		\item $R_1(t, s) = \min\{t, s\} - ts$,
		\item $R_2(t, s) = \min\{t, s\} - t(s + 1).$
	\end{enumerate}

\item Заданы случайные величины $v_1, v_2, u_1, u_2$ 
такие, что $\Expect v_i = \Expect u_i = 0,$
$\Expect v^2_i = 1, \Expect u^2_i = 4, i=1,2,$ 
а нормированная корреляционная матрица системы 
$(v_1, v_2, u_1, u_2)$ равна
\[
\begin{bmatrix}
    1 & 0 & \frac{1}{2} & 0 \\
    0 & 1 & 0 & -\frac{1}{2} \\
    \frac{1}{2} & 0 & 1 & 0 \\
    0 & -\frac{1}{2} & 0 & 1
\end{bmatrix}.
\]
Для случайных процессов
\begin{align*}
X_t = v_1 \cos \omega_1 t + v_2 \sin \omega_1 t, \\
Y_t = u_1 \cos \omega_2 t + u_2 \sin \omega_2 t
\end{align*}
найти взаимные корреляционные функции 
$r_{XY}(t_1, t_2) = \mathrm{corr}(X_{t_1}, Y_{t_2})$
и $r_{YX}(t_1, t_2) = \mathrm{corr}(Y_{t_1}, X_{t_2})$
и вычислить их значения при $t_1 = 0, t_2 = 1.$

\item Пусть $X = (X_1, \ldots, X_n)$ -- случайный вектор.
Докажите эквивалентность следующих утверждений
(в обе стороны):
	\begin{enumerate}
		\item характеристическая функция вектора $X$ допускает представление
		\begin{equation*}
	    \varphi_X(\mathbf{u}) = \exp\big\{
	    	i \mathbf{u}^{\intercal} \mu - 
	    		\frac 1 2 \mathbf{u}^{\intercal} \Sigma \mathbf{u} 
	    	\big\},
		\end{equation*}
		где $\mu$ -- неслучайный вектор из $\mathbb{R}^n$,
		а $\Sigma$ -- симметричная неотрицательно определенная неслучайная
		матрица размера $n \times n$,

		\item вектор $X$ допускает представление
		\begin{equation*}
		    X = \mu + AZ,
		\end{equation*}
		где $\mu$ -- неслучайный вектор из $\mathbb{R}^n$,
		$A$ -- неслучайная матрица размера $n \times n$,
		а $Z \in \mathbb{R}^n$ -- вектор, все координаты 
		которого независимы в совокупности и имеют нормальное $\mathcal{N}(0, 1)$
		распределение.
	\end{enumerate}

\item Пусть $\newprocess{B}$ -- винеровский процесс. Доказать,
что следующие процессы также винеровские:
\begin{enumerate}
	\item $B_t^{(1)} = 
	\begin{cases}
		0, \quad & t = 0, \\
		t B_{1/t}, \quad & t > 0.
	\end{cases}$
	\item $B_t^{(2)} = 
	\sqrt{c} B_{t/c}, \qquad c = \mathrm{const} > 0$.
\end{enumerate}

\item $\xi_1, \ldots, \xi_n$ -- независимые одинаково распределенные 
показательные случайные величины.
Подсчитать (по индукции) плотность распределения суммы
$\xi_1 + \ldots + \xi_n$.

\item Пусть $\newprocess{N}$ -- пуассоновский случайный
процесс с параметром $\lambda$. Доказать, что случайный
процесс $\newprocess{M}$, задаваемый соотношением
$M_t = N_{t+1} - N_t$, является стационарным второго порядка
процессом, т.е. что его математическое ожидание $\Expect M_t$
не зависит от времени, а его ковариационная функция
$R_M(t_1, t_2)$ зависит от $t_1$ и $t_2$ через их разность
$\tau = t_1 - t_2$.

\item Стандартное фрактальное броуновское движение $\newprocessfc{\fbm}{T}$
на $[0,T]$ с параметром Хёрста $H \in (0,1)$~---
это гауссовский процесс с~непрерывными траекториями такой, что
\[
\fbm_t = \frac{1}{\Gamma(H + \frac 1 2)}
	\int\limits_{-\infty}^{0} [(t - s)^{H - \frac 1 2 } - (-s)^{H-\frac 1 2}] dB_s
	+ \int\limits_{0}^{t} (t - s)^{H - \frac 1 2 } dB_s,
\]
где 
$\Gamma(\alpha) = \int_0^{\infty} x^{\alpha - 1} e^{-x} dx$ -- гамма-функция Эйлера.
Смоделируйте реализации фрактального броуновского
движения с помощью вычисления стохастического интеграла
по броуновскому движению. В качестве результата
приведите:
\begin{enumerate}
	\item разностную схему, использовавшуюся для моделирования,
	\item исходный код, использовавшийся для моделирования,
	\item примеры траекторий фрактального броуновского движения
	для различных значений параметра Херста $H \in (0, 1)$.
\end{enumerate}


\end{enumerate}


\newpage 

\section*{Вариант 2}


\begin{enumerate}

\item Доказать положительную определенность следующих функций:
	\begin{enumerate}
		\item $R_1(t, s) = 
			\begin{cases}
				1 - |t - s|, \quad & |t - s| < 1, \\
				0, \quad & |t - s| >= 1.
			\end{cases}$,
		\item $R_2(t, s) = e^{- |t - s|}$.
	\end{enumerate}

\item Случайный процесс $\newprocess{X}$ имеет вид
\[
X_t = b \sin (\gamma t + \varphi),
\]
где $b, \gamma$ -- известные постоянные, а $\varphi$ -- 
случайная величина с плотностью $f_{\varphi}(x)$. Исследовать
процесс $X$ на стационарность в узком и широком смысле,
а также на эргодичность по математическому ожиданию,
если 
	\begin{enumerate}
		\item $f_{\varphi}(x) = \cos(x) \mathds{1}_{[0, \frac \pi 2]}(x)$,
		\item $f_{\varphi}(x) = \frac{1}{2\pi} \mathds{1}_{[0, 2 \pi]}(x)$.
	\end{enumerate}

\item Доказать эквивалентность следующих двух определений
винеровского процесса (доказательство провести в обе стороны):
	\begin{enumerate}
		\item винеровский процесс -- это гауссовский
		процесс $\newprocess{B}$ с математическим ожиданием
		$\Expect B_t \equiv m(t) = 0$ и ковариационной
		функцией $\Expect (B_s - \Expect B_s) (B_t - \Expect B_t)
		\equiv R(s, t) = min\{s, t\}$,
		\item винеровский процесс -- это случайный процесс $\newprocess{B}$ такой, что
		\begin{itemize}
    		\item $B_0 = 0$ п.н.,
    		\item $B$ -- процесс с независимыми приращениями,
    		\item $B_t - B_s \sim \mathcal{N}(0, t - s) \quad \forall t > s \geqslant 0$.
		\end{itemize}
	\end{enumerate}

\item Пусть $\newprocess{B}$ -- винеровский процесс на $[0, t]$. Подсчитать 
	\begin{enumerate}
	    \item $\lim_{n \to \infty} \sum_{i=1}^n \left(W_{t_i} - W_{t_{i-1}}\right)^2$,
	    \item $\lim_{n \to \infty} \sum_{i=1}^n \left|W_{t_i} - W_{t_{i-1}}\right|$.
	\end{enumerate}

\item Пусть $\newprocess{N}$ -- неоднородный пуассоновский
процесс с интенсивностью $\lambda(t)$. Доказать, что 
	\begin{enumerate}
	    \item функция $\Lambda(t) = \int_{0}^t \lambda(s) \, ds$ имеет обратную,
	    \item процесс $M_t = N_{\Lambda^{-1}(t)}$ является однородным пуассоновским процессом. 
	\end{enumerate}

\item Пользовательские запросы поступают на веб-сервис
в соответствии с однородным пуассоновским потоком $\newprocess{N}$
с интенсивностью $\lambda$. Сервис оснащен балансировщиком
нагрузки, разделяющим запросы на $r$ подпотоков
$\{X^i\}_{i = 1}^r$ ($N_t = \sum_{i=1}^r X^i_t$)
таким образом, что каждый запрос из $N_t$ 
относится к подпотоку $X^i_t$ с вероятностью
$p_i, i=1, \ldots, r$ (независимо от других событий).
Определить тип и параметры случайных процессов
$\{X^i\}_{i = 1}^r$.

\item Составной пуассоновский поток событий (или пакетный пуассоновский процесс)
-- это случайный процесс $\newprocess{X}$ со скачками в моменты
скачков пуассоновского потока с заданной интенсивностью $\lambda$
и являются случайными величинами с заданным распределением $G$,
не зависящими от пуассоновского потока. Он может быть записан в виде:
\[
X_t = \sum\limits_{i = 1}^{N_t} D_i,
\]
где $X_t$ -- значение составного потока в момент $t$, $N_t$ -- 
значение простого потока в момент $t$ (число появлений пакетов),
и $D_i, i \geqslant 1$ -- последовательность независимых
одинаково распределенных случайных величин, имеющих распределение $G$.
Смоделируйте реализации этого процесса, если
	\begin{itemize}
		\item $\lambda = \lambda(t) = 
		2 + sin (t - 11\pi/16) + sin (2t - 3\pi/8)$, и
		\item $G(x)$ -- распределение Пуассона с параметром $\rho > 0$.
	\end{itemize}
В качестве результата приведите:
\begin{enumerate}
	\item исходный код, использовавшийся для моделирования,
	\item примеры траекторий составного пуассоновского потока
	для различных значений параметра $\rho$.
\end{enumerate}


\end{enumerate}

\end{document}
