% !TEX encoding = UTF-8 Unicode
\documentclass[a4paper,14pt]{extreport}
\usepackage[utf8]{inputenc}
\usepackage{color}
\usepackage[T2A]{fontenc}
\usepackage[intlimits]{amsmath}
\usepackage{amssymb}
\usepackage{dsfont}
\usepackage{bm}
\usepackage{indentfirst}
\usepackage{fullpage}
\usepackage{soul}
\usepackage[normalem]{ulem}
\usepackage[english,russian]{babel}
\usepackage{hyperref}
\hypersetup{
    colorlinks=true,
    linkcolor=blue,
    filecolor=magenta,      
    urlcolor=cyan,
}
\usepackage{mathtools}
\usepackage{graphicx}
\graphicspath{{Pictures/}}

\linespread{1.33}                    % Полуторный интвервал (ГОСТ Р 7.0.11-2011, 5.3.6)
% \Large


% \renewcommand{\labelenumi}{(\alph{enumi})} % Use letters for enumerate
% \DeclareMathOperator{\Sample}{Sample}
\let\vaccent=\v % rename builtin command \v{} to \vaccent{}
\usepackage{enumerate}
\renewcommand{\v}[1]{\ensuremath{\mathbf{#1}}} % for vectors
\newcommand{\gv}[1]{\ensuremath{\mbox{\boldmath$ #1 $}}} 
% for vectors of Greek letters
\newcommand{\uv}[1]{\ensuremath{\mathbf{\hat{#1}}}} % for unit vector
\newcommand{\abs}[1]{\left| #1 \right|} % for absolute value
\newcommand{\avg}[1]{\left< #1 \right>} % for average
\let\underdot=\d % rename builtin command \d{} to \underdot{}
\renewcommand{\d}[2]{\frac{d #1}{d #2}} % for derivatives
\newcommand{\dd}[2]{\frac{d^2 #1}{d #2^2}} % for double derivatives
\newcommand{\pd}[2]{\frac{\partial #1}{\partial #2}} 
% for partial derivatives
\newcommand{\pdd}[2]{\frac{\partial^2 #1}{\partial #2^2}} 
% for double partial derivatives
\newcommand{\pdc}[3]{\left( \frac{\partial #1}{\partial #2}
 \right)_{#3}} % for thermodynamic partial derivatives
\newcommand{\ket}[1]{\left| #1 \right>} % for Dirac bras
\newcommand{\bra}[1]{\left< #1 \right|} % for Dirac kets
\newcommand{\braket}[2]{\left< #1 \vphantom{#2} \right|
 \left. #2 \vphantom{#1} \right>} % for Dirac brackets
\newcommand{\matrixel}[3]{\left< #1 \vphantom{#2#3} \right|
 #2 \left| #3 \vphantom{#1#2} \right>} % for Dirac matrix elements
\newcommand{\grad}[1]{\gv{\nabla} #1} % for gradient
\let\divsymb=\div % rename builtin command \div to \divsymb
\renewcommand{\div}[1]{\gv{\nabla} \cdot \v{#1}} % for divergence
\newcommand{\curl}[1]{\gv{\nabla} \times \v{#1}} % for curl
\let\baraccent=\= % rename builtin command \= to \baraccent
\renewcommand{\=}[1]{\stackrel{#1}{=}} % for putting numbers above =
\providecommand{\wave}[1]{\v{\tilde{#1}}}
\providecommand{\fr}{\frac}
\providecommand{\RR}{\mathbb{R}}
\providecommand{\NN}{\mathbb{N}}
\providecommand{\seq}{\subseteq}
\providecommand{\e}{\varepsilon}

% LaTeX internal stuff
\newcommand{\innermid}{\nonscript\;\delimsize\vert\nonscript\;}
\newcommand{\activatebar}{%
  \begingroup\lccode`\~=`\|
  \lowercase{\endgroup\let~}\innermid 
  \mathcode`|=\string"8000
}

% Probability theory
\newcommand{\Expect}{\mathop{{}\mathrm{E}}}
\newcommand{\Expectmore}{\operatorname{E}\expectarg}
\newcommand{\covmore}{\operatorname{cov}\expectarg}
\DeclarePairedDelimiterX{\expectarg}[1]{[}{]}{%
  \ifnum\currentgrouptype=16 \else\begingroup\fi
  \activatebar#1
  \ifnum\currentgrouptype=16 \else\endgroup\fi
}
\newcommand{\Proba}{\mathrm{P}}
\newcommand{\Var}{\mathop{{}\mathrm{V}}}

% Stochastic processes
\newcommand{\generaltime}{t \geqslant 0}
\newcommand{\discretetime}{t = 1,\,2,\,\ldots}
\newcommand{\conttime}{t \in \realline^{+}}
\newcommand{\finiteconttime}[1]{0 \leqslant t \leqslant #1}
\newcommand{\finitediscretetime}[1]{\running{t}{1}{#1}}

\newcommand{\newprocess}[1]{
	\ensuremath{
		#1 = \left(#1 _t\right)_{\generaltime}
	}
}
\newcommand{\newprocessd}[1]{
	\ensuremath{
		#1 = \left(#1 _t\right)_{\discretetime}
	}
}
\newcommand{\newprocessfc}[2]{
	\ensuremath{
		#1 = \left(#1 _t\right)_{\finiteconttime{#2}}
	}
}

\newcommand{\trajectory}[1]{\big\{#1_s,\,0\leqslant s \leqslant t\big\}}
\newcommand{\firsttime}[1]{\ensuremath{\inf \{\generaltime: #1\}}}

\newcommand{\filtration}[1]{\mathcal{#1}}
\newcommand{\filtrationprocess}[2]{\filtration{#1}^{#2}}
\newcommand{\filtrationflow}[1]{\newprocess{\filtration{F}}}

\newcommand{\fbm}{B^H}

\newenvironment{boenumerate}
  {\begin{enumerate}\renewcommand\labelenumi{\textbf{\theenumi.}}}
  {\end{enumerate}}


\begin{document}


\begin{center}
    \begin{tabular}{|p{15.5cm}|}
        \hline
        \textbf{ФКН ВШЭ, 3 курс, 3 модуль}\\
        \begin{center} \Large Задание 2. Марковские цепи. Авторегрессионные
        и условно-гауссовские модели временных рядов
        \end{center}\\
        \textbf{Вероятностные модели и статистика случайных процессов, весна 2017}\\
        \hline
    \end{tabular}
\end{center}

Время выдачи задания: 10 февраля (пятница).

Срок сдачи: \textcolor{blue}{\bf 24 февраля (пятница), 23:59.}

Среда для выполнения практического задания~-- PYTHON 2.x.

\section*{Правила сдачи}

\textbf{Инструкция по отправке:} 
\begin{enumerate}
\item Домашнее задание необходимо отправить
до дедлайна на почту \href{mailto:hse.cs.stochastics@gmail.com}
{hse.cs.stochastics@gmail.com}.
\item В письме укажите тему 
<<[ФКН ССП17] Задание 2, Фамилия Имя>>.
\item Решения задач следует присылать единым файлом
формата .pdf, набранным в \LaTeX. Допускается отправка
отдельных практических задач в виде отдельных файлов (ipython-тетрадок
или исходных файлов с кодом на языке \texttt{python}).
\end{enumerate}

\textbf{Оценивание и штрафы:} 
\begin{enumerate}
\item \textbf{Каждая из задач имеет стоимость 2 балла}, при этом
за задачу можно получить 0, 1 или 2 балла.
Максимально допустимая оценка за работу -- 10 баллов. 
Баллы, набранные сверх максимальной оценки, 
считаются бонусными и влияют на освобождение от задач на экзамене.
\item Дедлайн жесткий. Сдавать задание после указанного
срока сдачи нельзя.
\item Задание выполняется самостоятельно.
<<Похожие>> решения считаются плагиатом и все задействованные
студенты (в том числе те, у кого списали) не могут получить
за него больше 0 баллов (подробнее о плагиате см. на странице курса).
Если вы нашли решение какого-то из заданий (или его часть)
в открытом источнике, необходимо указать ссылку на этот источник
в отдельном блоке в конце Вашей работы (скорее всего вы будете не
единственным, кто это нашел, поэтому чтобы исключить подозрение
в плагиате, необходима ссылка на источник).
\end{enumerate}


\newpage

\section*{Необходимые теоретические сведения}

\begin{boenumerate}
\item Случайный процесс $\newprocess{X}$ с дискретным
множеством $E$ значений и матрицей $P$ переходных вероятностей
называется дискретной марковской цепью, если:
	\begin{enumerate}
	\item $P = (p_{ij}), \qquad
	\sum\limits_{j \in E} p_{ij} = 1, \qquad
	p_{ij} \geqslant 0, \qquad i, j \in E$,
	\item $\Proba (X_n = j | X_{n-1} = i) = p_{ij}, \quad i, j \in E, \quad n \in \mathbb{N}$,
	\item $\Proba (X_{n+1} = j | X_0 = i_0, \ldots, X_n = i_n) = 
		\Proba (X_{n+1} = j | X_n = i_n).$
	\end{enumerate}

\item Процесс скользящего среднего MA($q$) -- это процесс $\newprocess{X}$,
задаваемый уравнением
\[
X_t = b_1 \varepsilon_{t-1} + \ldots + b_q \varepsilon_{t-q} + \sigma^2 \varepsilon_t,
\]
где параметры $b_i \in \mathbb{R}, i=1, \ldots, q$, $\sigma^2 > 0$,
а процесс $\newprocess{\varepsilon}$ -- последовательность 
независимых одинаково стандартно нормально распределенных случайных величин.

\item Авторегрессионный процесс AR($p$) -- это процесс $\newprocess{X}$,
задаваемый уравнением
\[
X_t = a_0 + a_1 X_{t-1} + \ldots + a_p X_{t-p} + \sigma^2 \varepsilon_t,
\]
где параметры $a_i \in \mathbb{R}, i=0, \ldots, p$, $\sigma^2 > 0$,
а процесс $\newprocess{\varepsilon}$ -- последовательность 
независимых одинаково стандартно нормально распределенных случайных величин.

\item Пусть $\newprocess{X}$ -- процесс авторегрессии.
Уравнения Юла-Уолкера выражают коэффициенты автоковариации
с заданным {\em лагом} $k$ (т.\,е. величины $R(k) = \Expectmore {X_t X_{t+k}}$)
процесса $\newprocess{X}$ через коэффициенты автоковариации с меньшими лагами:
\[
R(k) = a_1 R(k - 1) + \ldots + a_p R(k - p),
\quad
k \in \mathbb{N}.
\]
Аналогичное соотношение справедливо и для коэффициентов
{\em корреляции} $\rho_k = \rho(k) = R(k) / R(0)$, где $R(0)$ --
дисперсия временного ряда $X_t$.


\item Модель ARMA($p, q$) (autoregressive moving average) -- 
это стохастическая модель временного ряда, в которой наблюдения $\newprocess{X}$
задаются соотношением
\[
X_t = a_0 + \sum\limits_{i=1}^p a_i X_{t-i} 
	+ \sum\limits_{i=1}^p b_i \varepsilon_{t-i} + \sigma \varepsilon_t,
\]
где имеются члены AR($p$) и MA($q$).

\item Модель ARMAX($p, q, r$) (autoregressive moving average with exogenous inputs) -- 
это стохастическая модель временного ряда, в которой наблюдения $\newprocess{X}$
задаются соотношением
\[
X_t = a_0 + \sum\limits_{i=1}^p a_i X_{t-i} 
	+ \sum\limits_{i=1}^p b_i \varepsilon_{t-i} + \sigma \varepsilon_t
	+ \sum\limits_{i=1}^r c_i u_{t-i},
\]
где имеются члены AR($p$), MA($q$), и
$\newprocess{u}$ -- это некоторая заданная (возможно, случайная)
последовательность.

\item Модель ARCH($p$) (autoregressive conditional heteroscedasticity) -- 
это стохастическая модель временного ряда, в которой наблюдения $\newprocess{X}$
задаются соотношением
\[
X_t = \sigma_t \varepsilon_t,
\qquad
\sigma^2_t = \alpha_0 + \alpha_1 X^2_{t-1} \ldots + \alpha_p X^2_{t-p},
\]
где параметры $\alpha_i \in \mathbb{R}, i=0, \ldots, p$,
а процесс $\newprocess{\varepsilon}$ -- последовательность 
независимых одинаково стандартно нормально распределенных случайных величин.

\item Модель GARCH($p, q$) (generalized autoregressive conditional heteroscedasticity) -- 
это стохастическая модель временного ряда, в которой наблюдения $\newprocess{X}$
задаются соотношением
\[
X_t = \sigma_t \varepsilon_t,
\qquad
\sigma^2_t = \alpha_0 + \sum\limits_{i = 1}^p\alpha_i X^2_{t-i} + \sum\limits_{i = 1}^q \beta_i \sigma^2_{t-i},
\]
где параметры $\alpha_i, \beta_j \in \mathbb{R},
 i=0, \ldots, p, j = 1, \ldots, q$,
а процесс $\newprocess{\varepsilon}$ -- последовательность 
независимых одинаково стандартно нормально распределенных случайных величин.

\item Оператор сдвига индекса временного ряда $L$ -- это оператор, изменяющий
индекс временного ряда на меньший согласно соотношению
\[
LX_{t} = X_{t - 1}, \quad t \in \mathbb{Z}.
\]

% \item Передаточной функцией авторегрессионного процесса AR($p$),
% задаваемого уравнением 
% \[
% X_t = a_0 + a_1 X_{t-1} + \ldots + a_p X_{t-p} + \sigma^2 \varepsilon_t,
% \]
% который может быть записан в эквивалентном виде
% \[
% X_t = \sigma^2 \varepsilon_t + ,
% \]
% называется функция $\theta(B)$, задаваемая 
\end{boenumerate}


\newpage


\section*{Вариант 1}


\begin{enumerate}

	\item Пусть $\xi_1, \xi_2, \ldots$ -- независимые одинаково
	распределенные случайные величины, причем
	\[
	\Proba (\xi_t = 1) = p, \quad  \Proba (\xi_t = -1) = 1 - p.
	\]
	Проверить, являются ли цепями Маркова следующие последовательности
	случайных величин:
	\begin{enumerate}
		\item $\eta_t = \xi_t \xi_{t + 1}$,
		\item $\eta_t = \xi_1 \xi_2 \ldots \xi_t$.
	\end{enumerate}
	Для цепей Маркова найти вероятности перехода за один шаг.

	\item Движение частицы по целым точкам отрезка $[0, N]$
	описывается цепью Маркова с $N+1$ состояниями и вероятностями
	перехода за один шаг $p_{00} = p_{NN} = 1$; $p_{i, i+1} = p, p_{i, i-1} = 1 - p = q$;
	$i = 1, \ldots, N - 1$ (<<случайное блуждание с поглощающими
	стенками>>). Пусть  $\xi_t$ -- положение частицы 
	в момент $t$, и $\tau = \min \{t: \xi_t = N\}$~-- первый момент
	времени частицей верхней стенки. Показать, что 
	\[
	m_k = \Expectmore {\tau | \xi_0 = k} = 
	\begin{cases}
		\frac{2pq}{(p - q)^2} 
			\Big( \Big(\frac q p\Big)^N - \Big(\frac q p\Big)^k \Big) 
			- \frac{N - k}{q - p}, & \mbox{если } p \neq q,\\
			(N - k)(N + k), & \mbox{если } p = q.
	\end{cases}
	\]

	% \item Используя предлагаемый в ipython-тетрадке \texttt{mc-music.ipynb} код,
	% напишите программу для генерации музыкальных импровизаций с помощью марковской цепи.
	% Следуйте инструкциям, изложенным в тетрадке. В качестве ответа приложите
	% модифицированную тетрадку, содержащую ваш код и сгенерированную последовательность нот.

	\item Для случайного процесса $\newprocess{h}$, заданного выражением:
	\begin{enumerate}
		\item $h_t = \varepsilon_t - 1.3 \varepsilon_{t-1} + 0.4 \varepsilon_{t-2}$,
		\item $h_t - h_{t-1} = \varepsilon_t - 1.3 \varepsilon_{t-1} + 0.3 \varepsilon_{t-2}$,
	\end{enumerate}
	где $\newprocess{\varepsilon}$ -- последовательность независимых
	одинаково стандартно нормально распределенных случайных величин,
	\begin{enumerate}
	\item Записать эквивалентные представления с использованием оператора сдвига $L$;
	\item Провести исследование на стационарность 2-го порядка (с доказательством);
	\item Вычислить первые четыре автокорреляции $\rho_0, \rho_1, \rho_2, \rho_3$;
	\item Вычислить дисперсию случайной величины $h_t$;
	\item Определить тип процесса в терминах ARMA($p, q$);
	\item Записать уравнения Юла-Уолкера;
	\item Решить эти уравнения с определением коэффициентов автокорреляции $\rho_1, \rho_2$.
	% \item получить частичную автокорреляционную функцию.
	\end{enumerate}

	\item Рассматривается процесс ARMAX, заданный уравнением
	\[
	X_t - 1.5 X_{t-1} + 0.7 X_{t-2} = u_{t-1} + 0.5 u_{t-2} + 
		\varepsilon_{t} - \varepsilon_{t-1} + 0.2 \varepsilon_{t-2},
	\]
	где $\newprocess{\varepsilon}$ -- процесс белого шума с дисперсией 0.25,
	последовательность $\newprocess{u}$ -- случайная бинарная ($\pm 1$) последовательность.
	\begin{enumerate}
		\item Сгенерировать траекторию длины $N = 500$ указанного процесса;
		\item Выписать функционал правдоподобия указанного процесса
		и выражения для оценок максимального правдоподобия его параметров
		в предположении, что известны порядки частей AR, MA и X процесса;
		\item По сгенерированной выборке оценить параметры
		процесса ARMAX в предположении, что полностью известна
		модель (известны порядки частей AR, MA и X процесса);
		\item Построить графики зависимости оценок параметров
		процесса ARMAX (предполагается, что полностью известна
		модель процесса) от объема использованной выборки.
		Сходятся ли эти оценки к настоящим значениям?
		% \item Подогнать к сгенерированной траектории процесс
		% авторегрессии порядка $p = 4$. Сравнить передаточную
		% функцию (нарисовать передаточные функции на одном графике)
		% полученного фильтра авторегрессии с передаточной
		% функцией фильтра исходного процесса ARMAX.
	\end{enumerate}

	\item 
	Доходность акций часто оценивают в процентах согласно следующей формуле.
	Если $S_n$ -- цена закрытия акции в день $n, n \geqslant 1$, то ее доходность равна величине
	$X_n = \frac{S_n - S_{n-1}}{S_n}$ (отношение прироста цены закрытия
	за текущий день к цене предыдущего дня, daily returns in percent). 

	Для моделирования временного ряда, соответствующего процентной
	доходности некоторого фондового индекса, использовалось 1000 наблюдений,
	к которым была подогнана GARCH-модель. Результатом подгонки являются уравнения:
	\begin{align*}
	X_n = 0.102 + h_n, 
	\quad 
	h_n = \sigma_n \varepsilon_n, \\
	\sigma_n^2 = 0.146 + 0.1068 h_{n-1}^2 + 0.8212 \sigma_{n-1}^2.
	\end{align*}
	Здесь $X_n$ -- доходность в день $n$, $\newprocessd{\varepsilon}$ --
	процесс гауссовского белого шума.

	Доходность в день $n = 1000$ (2013-05-31) составляла
	$X_{1000} = -1.354$, причем $\sigma^2_{1000} =  2.27$.
	В день $n + 1 = 1001$ (2013-06-03)
	наблюдаемая доходность составила $X_{1001} = -10.47$,
	в то время как прогноз дисперсии в этот день составил $2.24$.

	\begin{enumerate}
	\item Записать формулу для подсчета $\sigma^2_{1001}$, 
	подставить числовое значение, подсчитать результат;
	\item Подсчитать распределение доходности $X_{1001}$;
	\item Обосновать ожидаемость или невозможность наблюдений
	фактической доходности $X_{1000} = -1.354$;
	\item Используя приведенный пример, обосновать утверждение
	о том, что GARCH является моделью с условной дисперсией;
	\item Используя уравнения модели, охарактеризовать
	{\em безусловную} дисперсию доходности.
	\end{enumerate}

	\item Вам выдан файл \texttt{aapl.txt}, содержащий
	значения наблюдаемой доходности
	$X_n = \frac{S_n - S_{n-1}}{S_n}$ акций компании Apple 
	в период с 01.01.2007 по 01.01.2017.

	\begin{enumerate}
	\item Записать формулу для правдоподобия условно-гауссовской модели
	ARCH($p$);
	\item Используя несколько различных значений $p$,
	оценить параметры модели ARCH($p$),
	прокомментировать качество оценки для различных $p$;
	\item Определить, характеризуются ли данные кластерами волатильности
	(volatility clustering), т.е. дает ли модель ARCH($p$) 
	преимущества по сравнению с однородной моделью,
	для которой $\sigma^2_n = \mathrm{const}$;
	\item Для выбранного вами значения $p$ нарисовать график
	волатильности $\sigma^2_n$;
	\item Описать процедуру выбора оптимального значения $p$
	с помощью какого-либо информационного критерия.
	\end{enumerate}


\end{enumerate}


\newpage 

\section*{Вариант 2}


\begin{enumerate}

	\item Пусть $\xi_1, \xi_2, \ldots$ -- независимые одинаково
	распределенные случайные величины, причем
	\[
	\Proba (\xi_t = 1) = p, \quad \Proba (\xi_t = -1) = 1 - p = q.
	\]
	Положим $\eta_0 = 0, \eta_{t+1} = \eta_t + \xi_{t + 1}$. 
	Выяснить, является ли последовательность $\eta_t$ цепью Маркова. 
	Найти $\Proba(\eta_t = m), m = 0, 1, \ldots$

	\item Движение частицы по целым точкам отрезка $[0, N]$
	описывается цепью Маркова с $N+1$ состояниями и вероятностями
	перехода за один шаг $p_{00} = p_{NN} = 1$; $p_{i, i+1} = p, p_{i, i-1} = 1 - p = q$;
	$i = 1, \ldots, N - 1$ (<<случайное блуждание с поглощающими
	стенками>>). Пусть  $\xi_t$ -- положение частицы 
	в момент $t$, и $p_{ij}(t) = \Proba (\xi_t = j | \xi_0 = i)$.
	Найти вероятности поглощения частицы в точках $0$ и $N$:
	\[
	\pi_k^{(0)} = \lim\limits_{t \to \infty} p_{k0}(t),
	\qquad
	\pi_k^{(N)} = \lim\limits_{t \to \infty} p_{kN}(t).
	\]

	% \item Используя предлагаемый в ipython-тетрадке \texttt{mc-text.ipynb} код,
	% напишите программу для генерации текста с помощью марковской цепи.
	% Следуйте инструкциям, изложенным в тетрадке. В качестве ответа приложите
	% модифицированную тетрадку, содержащую ваш код и сгенерированный текст.

	\item Для случайного процесса $\newprocessd{h}$, заданного выражением:
	\begin{enumerate}
		\item $h_t - 0.5 h_{t-1} = \varepsilon_t - 1.3 \varepsilon_{t-1} + 0.4 \varepsilon_{t-2}$,
		\item $h_t - 1.5 h_{t-1} + 0.6 h_{t-2} = \varepsilon_t$,
	\end{enumerate}
	где $\newprocess{\varepsilon}$ -- последовательность независимых
	одинаково стандартно нормально распределенных случайных величин,
	\begin{enumerate}
	\item Записать эквивалентные представления с использованием оператора сдвига $L$;
	\item Провести исследование на стационарность 2-го порядка (с доказательством);
	\item Вычислить первые четыре автокорреляции $\rho_0, \rho_1, \rho_2, \rho_3$;
	\item Вычислить дисперсию случайной величины $h_t$;
	\item Определить тип процесса в терминах ARMA($p, q$);
	\item Записать уравнения Юла-Уолкера;
	\item Решить эти уравнения с определением коэффициентов автокорреляции $\rho_1, \rho_2$.
	% \item получить частичную автокорреляционную функцию.
	\end{enumerate}

	\item Сгенерировать траекторию длины $N = 80$ авторегрессионного процесса
	\[
	(1 + 1.5 L^{-1} + 0.5625 L^{-2}) X_t = 0.1 \varepsilon_t,
	\]
	где $\newprocess{\varepsilon}$ -- последовательность 
 	независимых одинаково стандартно нормально распределенных случайных величин.
	\begin{enumerate}
	\item Подогнать к сгенерированной траектории процессы
	скользящего среднего MA($q$) порядков $q \in \{1, 2, 3, 4\}$.
	Какая из этих моделей, согласно критерию AIC, является 
	наиболее подходящей для моделирования сгенерированного временного ряда?
	
	% Сравнить
	% передаточные функции (нарисовать передаточные функции на одном графике)
	% полученных фильтров скользящего среднего с передаточной функцией
	% фильтра заданного процесса авторегрессии, который использовался для генерации данных.

	\item Подогнать к сгенерированной траектории процессы авторегрессии AR($p$)
	порядков $p \in \{1,2,3,4\}$. Использовать для оценки параметров
	процессов авторегрессии метод максимального правдоподобия.
	Сравнить значения полученных оценок коэффициентов процесса
	авторегрессии со значениями оценок заданного процесса авторегрессии, который
	использовался для генерации данных.

	% Сравнить передаточные функции (нарисовать передаточные функции
	% на одном графике) полученных фильтров авторегрессии с передаточной
	% функцией фильтра заданного процесса авторегрессии, который
	% использовался для генерации данных.

	\item Подогнать к сгенерированной траектории процесс
	авторегрессии порядка $p = 2$ (то есть процесс авторегрессии
	правильного порядка), используя уравнения Юла-Уолкера (т.\,е. решая их
	относительно коэффициентов $a_0, a_1, \ldots, a_p$.
	Исследовать зависимость ошибки оценки параметров процесса
	авторегрессии от объема выборки.

	% Исследовать зависимость ошибки оценки параметров процесса
	% авторегрессии от объема выборки, сравнивая, например,
	% передаточные функции соответствующих фильтров.


	% \item получить частичную автокорреляционную функцию.
	\end{enumerate}


 

	\item Доходность акций часто оценивают в процентах согласно следующей формуле.
	Если $S_n$ -- цена закрытия акции в день $n, n \geqslant 1$, то ее доходность равна величине
	$X_n = \frac{S_n - S_{n-1}}{S_n}$ (отношение прироста цены закрытия
	за текущий день к цене предыдущего дня, daily returns in percent). 

	Для моделирования временного ряда, соответствующего процентной
	доходности некоторого фондового индекса, использовалось 1000 наблюдений,
	к которым была подогнана GARCH-модель. Результатом подгонки являются уравнения:
	\begin{align*}
	X_n = 0.102 + h_n, 
	\quad 
	h_n = \sigma_n \varepsilon_n, \\
	\sigma_n^2 = 0.146 + 0.1068 h_{n-1}^2 + 0.8212 \sigma_{n-1}^2.
	\end{align*}
	Здесь $X_n$ -- доходность в день $n$, $\newprocessd{\varepsilon}$ --
	процесс гауссовского белого шума.

	Доходность в день $n = 1000$ (2013-05-31) составляла
	$X_{1000} = -1.354$, причем $\sigma^2_{1000} =  2.27$.
	В день $n + 1 = 1001$ (2013-06-03)
	наблюдаемая доходность составила $X_{1001} = -10.47$,
	в то время как прогноз дисперсии в этот день составил $2.24$.

	\begin{enumerate}
	\item Записать формулу для подсчета $\sigma^2_{1001}$, 
	подставить числовое значение, подсчитать результат;
	\item Подсчитать распределение доходности $X_{1001}$;
	\item Обосновать ожидаемость или невозможность наблюдений
	фактической доходности $X_{1000} = -1.354$;
	\item Используя приведенный пример, обосновать утверждение
	о том, что GARCH является моделью с условной дисперсией;
	\item Используя уравнения модели, охарактеризовать
	{\em безусловную} дисперсию доходности.
	\end{enumerate}


	\item Вам выдан файл \texttt{goog.txt}, содержащий
	значения наблюдаемой доходности
	$X_n = \frac{S_n - S_{n-1}}{S_n}$ акций компании Google 
	в период с 01.01.2007 по 01.01.2017.

	\begin{enumerate}
	\item Записать формулу для правдоподобия условно-гауссовской модели
	GARCH($p, q$);
	\item Используя несколько различных значений $p$ и $q$,
	оценить параметры модели GARCH($p, q$),
	прокомментировать качество оценки для различных $p$ и $q$;
	\item Определить, характеризуются ли данные кластерами волатильности
	(volatility clustering), т.е. дает ли модель GARCH($p, q$) 
	преимущества по сравнению с однородной моделью,
	для которой $\sigma^2_n = \mathrm{const}$;
	\item Для выбранных вами значений $p,q$ нарисовать график
	волатильности $\sigma^2_n$;
	\item Описать процедуру выбора оптимального значения параметров $p$ и $q$
	с помощью какого-либо информационного критерия.
	\end{enumerate} 


\end{enumerate}

\end{document}
